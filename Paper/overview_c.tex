\begin{abstract}
随着计算机视觉技术的发展,对于实时视频源的计算变得越来越高效。位置信息的获取是计算机视觉领域中较重要的应用之一。室内的环境与室外环境有着诸多不同,其中墙壁对于GPS信号的阻碍作用使得基于卫星的定位在室内变得几乎不可能。因此,为了实现在室内的较精确的定位,需要引入一个全新的绝对参考系。本文提出了一种成本低廉,精度较高,部署方便的定位与追踪算法。
\end{abstract}
\newpage
\tableofcontents
\newpage
\section{总览}
室内的定位系统是近年来的研究热点之一。不少研究团队的研究领域在基于无线信号的定位系统,如基于wifi的室内定位系统\cite{cite1}。然而,由于室内的环境复杂,而无线信号基站的部署成本高昂,这样的定位系统不能得到有效应用。本文中提出了一种基于计算机视觉的定位系统。该系统中使用了普通的网络摄像头作为绝对参考系,从而使得部署成本大大降低。一些基于计算机算法的使用保证了该系统的健壮性与精确性。\\

其中使用的开源库有video4linux\cite{cite2}, OpenCV\cite{cite3}。在开发的过程中使用了Python这一灵活高效的编程语言,使得代码复杂度大大减小,而又不以牺牲较大的性能为代价。

此系统中,共有三个子系统,它们是:
\begin{enumerate}
  \item 视频提取
  \item 小孔摄像机数学模型
  \item Epipolar几何学
\end{enumerate}
